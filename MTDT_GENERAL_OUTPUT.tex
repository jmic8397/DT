% Options for packages loaded elsewhere
\PassOptionsToPackage{unicode}{hyperref}
\PassOptionsToPackage{hyphens}{url}
%
\documentclass[
]{article}
\usepackage{amsmath,amssymb}
\usepackage{lmodern}
\usepackage{ifxetex,ifluatex}
\ifnum 0\ifxetex 1\fi\ifluatex 1\fi=0 % if pdftex
  \usepackage[T1]{fontenc}
  \usepackage[utf8]{inputenc}
  \usepackage{textcomp} % provide euro and other symbols
\else % if luatex or xetex
  \usepackage{unicode-math}
  \defaultfontfeatures{Scale=MatchLowercase}
  \defaultfontfeatures[\rmfamily]{Ligatures=TeX,Scale=1}
\fi
% Use upquote if available, for straight quotes in verbatim environments
\IfFileExists{upquote.sty}{\usepackage{upquote}}{}
\IfFileExists{microtype.sty}{% use microtype if available
  \usepackage[]{microtype}
  \UseMicrotypeSet[protrusion]{basicmath} % disable protrusion for tt fonts
}{}
\makeatletter
\@ifundefined{KOMAClassName}{% if non-KOMA class
  \IfFileExists{parskip.sty}{%
    \usepackage{parskip}
  }{% else
    \setlength{\parindent}{0pt}
    \setlength{\parskip}{6pt plus 2pt minus 1pt}}
}{% if KOMA class
  \KOMAoptions{parskip=half}}
\makeatother
\usepackage{xcolor}
\IfFileExists{xurl.sty}{\usepackage{xurl}}{} % add URL line breaks if available
\IfFileExists{bookmark.sty}{\usepackage{bookmark}}{\usepackage{hyperref}}
\hypersetup{
  pdftitle={MTDT\_Function\_ClassifyR},
  pdfauthor={Andy Tran \& Jamie Mickaill},
  hidelinks,
  pdfcreator={LaTeX via pandoc}}
\urlstyle{same} % disable monospaced font for URLs
\usepackage[margin=1in]{geometry}
\usepackage{color}
\usepackage{fancyvrb}
\newcommand{\VerbBar}{|}
\newcommand{\VERB}{\Verb[commandchars=\\\{\}]}
\DefineVerbatimEnvironment{Highlighting}{Verbatim}{commandchars=\\\{\}}
% Add ',fontsize=\small' for more characters per line
\usepackage{framed}
\definecolor{shadecolor}{RGB}{248,248,248}
\newenvironment{Shaded}{\begin{snugshade}}{\end{snugshade}}
\newcommand{\AlertTok}[1]{\textcolor[rgb]{0.94,0.16,0.16}{#1}}
\newcommand{\AnnotationTok}[1]{\textcolor[rgb]{0.56,0.35,0.01}{\textbf{\textit{#1}}}}
\newcommand{\AttributeTok}[1]{\textcolor[rgb]{0.77,0.63,0.00}{#1}}
\newcommand{\BaseNTok}[1]{\textcolor[rgb]{0.00,0.00,0.81}{#1}}
\newcommand{\BuiltInTok}[1]{#1}
\newcommand{\CharTok}[1]{\textcolor[rgb]{0.31,0.60,0.02}{#1}}
\newcommand{\CommentTok}[1]{\textcolor[rgb]{0.56,0.35,0.01}{\textit{#1}}}
\newcommand{\CommentVarTok}[1]{\textcolor[rgb]{0.56,0.35,0.01}{\textbf{\textit{#1}}}}
\newcommand{\ConstantTok}[1]{\textcolor[rgb]{0.00,0.00,0.00}{#1}}
\newcommand{\ControlFlowTok}[1]{\textcolor[rgb]{0.13,0.29,0.53}{\textbf{#1}}}
\newcommand{\DataTypeTok}[1]{\textcolor[rgb]{0.13,0.29,0.53}{#1}}
\newcommand{\DecValTok}[1]{\textcolor[rgb]{0.00,0.00,0.81}{#1}}
\newcommand{\DocumentationTok}[1]{\textcolor[rgb]{0.56,0.35,0.01}{\textbf{\textit{#1}}}}
\newcommand{\ErrorTok}[1]{\textcolor[rgb]{0.64,0.00,0.00}{\textbf{#1}}}
\newcommand{\ExtensionTok}[1]{#1}
\newcommand{\FloatTok}[1]{\textcolor[rgb]{0.00,0.00,0.81}{#1}}
\newcommand{\FunctionTok}[1]{\textcolor[rgb]{0.00,0.00,0.00}{#1}}
\newcommand{\ImportTok}[1]{#1}
\newcommand{\InformationTok}[1]{\textcolor[rgb]{0.56,0.35,0.01}{\textbf{\textit{#1}}}}
\newcommand{\KeywordTok}[1]{\textcolor[rgb]{0.13,0.29,0.53}{\textbf{#1}}}
\newcommand{\NormalTok}[1]{#1}
\newcommand{\OperatorTok}[1]{\textcolor[rgb]{0.81,0.36,0.00}{\textbf{#1}}}
\newcommand{\OtherTok}[1]{\textcolor[rgb]{0.56,0.35,0.01}{#1}}
\newcommand{\PreprocessorTok}[1]{\textcolor[rgb]{0.56,0.35,0.01}{\textit{#1}}}
\newcommand{\RegionMarkerTok}[1]{#1}
\newcommand{\SpecialCharTok}[1]{\textcolor[rgb]{0.00,0.00,0.00}{#1}}
\newcommand{\SpecialStringTok}[1]{\textcolor[rgb]{0.31,0.60,0.02}{#1}}
\newcommand{\StringTok}[1]{\textcolor[rgb]{0.31,0.60,0.02}{#1}}
\newcommand{\VariableTok}[1]{\textcolor[rgb]{0.00,0.00,0.00}{#1}}
\newcommand{\VerbatimStringTok}[1]{\textcolor[rgb]{0.31,0.60,0.02}{#1}}
\newcommand{\WarningTok}[1]{\textcolor[rgb]{0.56,0.35,0.01}{\textbf{\textit{#1}}}}
\usepackage{graphicx}
\makeatletter
\def\maxwidth{\ifdim\Gin@nat@width>\linewidth\linewidth\else\Gin@nat@width\fi}
\def\maxheight{\ifdim\Gin@nat@height>\textheight\textheight\else\Gin@nat@height\fi}
\makeatother
% Scale images if necessary, so that they will not overflow the page
% margins by default, and it is still possible to overwrite the defaults
% using explicit options in \includegraphics[width, height, ...]{}
\setkeys{Gin}{width=\maxwidth,height=\maxheight,keepaspectratio}
% Set default figure placement to htbp
\makeatletter
\def\fps@figure{htbp}
\makeatother
\setlength{\emergencystretch}{3em} % prevent overfull lines
\providecommand{\tightlist}{%
  \setlength{\itemsep}{0pt}\setlength{\parskip}{0pt}}
\setcounter{secnumdepth}{-\maxdimen} % remove section numbering
\ifluatex
  \usepackage{selnolig}  % disable illegal ligatures
\fi

\title{MTDT\_Function\_ClassifyR}
\author{Andy Tran \& Jamie Mickaill}
\date{01/02/2022}

\begin{document}
\maketitle

\hypertarget{multi-tier-decision-tree-for-clinical-diagnostics}{%
\section{Multi Tier Decision Tree for Clinical
Diagnostics}\label{multi-tier-decision-tree-for-clinical-diagnostics}}

This program aims to use a sequential, machine learning approach to
build a forest of multi-level clinical diagnostic trees with multi-omics
data. This will allow for comparison of the cost and accuracy of
different diagnostic sequences. With this information, we hope to enable
better informed decisions of diagnostic processes.

At each node of the tree, a machine learning model is used to establish
the sample specific accuracy of health outcome predictions for a single
layer of the multi-omics data (e.g.~how accurate is the prediction for a
single individual based on their histology data). If the prediction can
not be made with sufficient accuracy, the individual will progress to
the next level where an alternative model and layer of the omics data
will be used. Currently each layer is analysed seperately, such that we
have analysis of multiple single-omics datasets (ensemble learning to be
implemented ?)

\hypertarget{demonstration-of-mtdt-function-for-classifyr}{%
\section{Demonstration of MTDT function for
ClassifyR}\label{demonstration-of-mtdt-function-for-classifyr}}

Below we create a mock MAE, assign some example runTest() parameters,
and pass it to our GeneralMTDT() function. We then generate plots with
the resulting data.

\hypertarget{mock-dataset}{%
\section{Mock Dataset}\label{mock-dataset}}

\begin{Shaded}
\begin{Highlighting}[]
\CommentTok{\#Mock Dataset}

\DocumentationTok{\#\#\#\#\#\#\#\#\#\#\#\#\#\#\#\#\#\#\#\#\#\#\#\#\#\#\#\#\#\#\#\#\#\#\#\#\#\#\#\#\#\#\#\#\#\#\#\#\#\#\#\#\#\#\#\#\#\#\#\#\#\#\#\#\#\#\#\#\#\#\#\#\#\#\#\#}

\NormalTok{  genesMatrix }\OtherTok{\textless{}{-}} \FunctionTok{matrix}\NormalTok{(}\FunctionTok{c}\NormalTok{(}\FunctionTok{rnorm}\NormalTok{(}\DecValTok{90}\NormalTok{, }\DecValTok{9}\NormalTok{, }\DecValTok{1}\NormalTok{),}
                         \FloatTok{5.2}\NormalTok{, }\FloatTok{5.3}\NormalTok{, }\FloatTok{9.4}\NormalTok{, }\FloatTok{9.4}\NormalTok{, }\FloatTok{9.6}\NormalTok{, }\FloatTok{9.9}\NormalTok{, }\FloatTok{9.1}\NormalTok{, }\FloatTok{9.8}\NormalTok{,}\FloatTok{9.5}\NormalTok{, }\FloatTok{9.4}\NormalTok{),}
              \AttributeTok{ncol =} \DecValTok{10}\NormalTok{, }\AttributeTok{byrow =} \ConstantTok{TRUE}\NormalTok{)}

\NormalTok{  genesMatrix2 }\OtherTok{\textless{}{-}} \FunctionTok{matrix}\NormalTok{(}\FunctionTok{c}\NormalTok{(}\FunctionTok{rnorm}\NormalTok{(}\DecValTok{90}\NormalTok{, }\DecValTok{9}\NormalTok{, }\DecValTok{1}\NormalTok{),}
                         \FloatTok{5.2}\NormalTok{, }\FloatTok{5.3}\NormalTok{, }\FloatTok{9.4}\NormalTok{, }\FloatTok{9.4}\NormalTok{, }\FloatTok{9.6}\NormalTok{, }\FloatTok{9.9}\NormalTok{, }\FloatTok{9.1}\NormalTok{, }\FloatTok{9.8}\NormalTok{,}\FloatTok{9.5}\NormalTok{, }\FloatTok{9.4}\NormalTok{),}
              \AttributeTok{ncol =} \DecValTok{10}\NormalTok{, }\AttributeTok{byrow =} \ConstantTok{TRUE}\NormalTok{)}

\NormalTok{    genesMatrix3 }\OtherTok{\textless{}{-}} \FunctionTok{matrix}\NormalTok{(}\FunctionTok{c}\NormalTok{(}\FunctionTok{rnorm}\NormalTok{(}\DecValTok{90}\NormalTok{, }\DecValTok{9}\NormalTok{, }\DecValTok{1}\NormalTok{),}
                         \FloatTok{5.2}\NormalTok{, }\FloatTok{5.3}\NormalTok{, }\FloatTok{9.4}\NormalTok{, }\FloatTok{9.4}\NormalTok{, }\FloatTok{9.6}\NormalTok{, }\FloatTok{9.9}\NormalTok{, }\FloatTok{9.1}\NormalTok{, }\FloatTok{9.8}\NormalTok{,}\FloatTok{9.5}\NormalTok{, }\FloatTok{9.4}\NormalTok{),}
              \AttributeTok{ncol =} \DecValTok{10}\NormalTok{, }\AttributeTok{byrow =} \ConstantTok{TRUE}\NormalTok{)}
  
  \FunctionTok{colnames}\NormalTok{(genesMatrix) }\OtherTok{\textless{}{-}} \FunctionTok{paste}\NormalTok{(}\StringTok{"Sample"}\NormalTok{, }\DecValTok{1}\SpecialCharTok{:}\DecValTok{10}\NormalTok{)}
  \FunctionTok{rownames}\NormalTok{(genesMatrix) }\OtherTok{\textless{}{-}} \FunctionTok{paste}\NormalTok{(}\StringTok{"Gene"}\NormalTok{, }\DecValTok{1}\SpecialCharTok{:}\DecValTok{10}\NormalTok{)}
    \FunctionTok{colnames}\NormalTok{(genesMatrix2) }\OtherTok{\textless{}{-}} \FunctionTok{paste}\NormalTok{(}\StringTok{"Sample"}\NormalTok{, }\DecValTok{1}\SpecialCharTok{:}\DecValTok{10}\NormalTok{)}
  \FunctionTok{rownames}\NormalTok{(genesMatrix2) }\OtherTok{\textless{}{-}} \FunctionTok{paste}\NormalTok{(}\StringTok{"Gene"}\NormalTok{, }\DecValTok{1}\SpecialCharTok{:}\DecValTok{10}\NormalTok{)}
      \FunctionTok{colnames}\NormalTok{(genesMatrix3) }\OtherTok{\textless{}{-}} \FunctionTok{paste}\NormalTok{(}\StringTok{"Sample"}\NormalTok{, }\DecValTok{1}\SpecialCharTok{:}\DecValTok{10}\NormalTok{)}
  \FunctionTok{rownames}\NormalTok{(genesMatrix3) }\OtherTok{\textless{}{-}} \FunctionTok{paste}\NormalTok{(}\StringTok{"Gene"}\NormalTok{, }\DecValTok{1}\SpecialCharTok{:}\DecValTok{10}\NormalTok{)}
  
\NormalTok{  genders }\OtherTok{\textless{}{-}} \FunctionTok{factor}\NormalTok{(}\FunctionTok{c}\NormalTok{(}\StringTok{"Male"}\NormalTok{, }\StringTok{"Male"}\NormalTok{, }\StringTok{"Female"}\NormalTok{, }\StringTok{"Female"}\NormalTok{, }\StringTok{"Female"}\NormalTok{,}
                    \StringTok{"Female"}\NormalTok{, }\StringTok{"Female"}\NormalTok{, }\StringTok{"Female"}\NormalTok{, }\StringTok{"Female"}\NormalTok{, }\StringTok{"Female"}\NormalTok{))}

  \CommentTok{\# Scenario: Male gender can predict the hard{-}to{-}classify Sample 1 and Sample 2.}
\NormalTok{  clinical }\OtherTok{\textless{}{-}} \FunctionTok{DataFrame}\NormalTok{(}\AttributeTok{Person\_ID =} \FunctionTok{c}\NormalTok{(}\DecValTok{1}\NormalTok{,}\DecValTok{2}\NormalTok{,}\DecValTok{3}\NormalTok{,}\DecValTok{4}\NormalTok{,}\DecValTok{5}\NormalTok{,}\DecValTok{6}\NormalTok{,}\DecValTok{7}\NormalTok{,}\DecValTok{8}\NormalTok{,}\DecValTok{9}\NormalTok{,}\DecValTok{10}\NormalTok{),}
                        \AttributeTok{class =} \FunctionTok{rep}\NormalTok{(}\FunctionTok{c}\NormalTok{(}\DecValTok{1}\NormalTok{, }\DecValTok{0}\NormalTok{), }\AttributeTok{each =} \DecValTok{5}\NormalTok{),}
    \AttributeTok{age =} \FunctionTok{c}\NormalTok{(}\DecValTok{31}\NormalTok{, }\DecValTok{34}\NormalTok{, }\DecValTok{32}\NormalTok{, }\DecValTok{39}\NormalTok{, }\DecValTok{33}\NormalTok{, }\DecValTok{38}\NormalTok{, }\DecValTok{34}\NormalTok{, }\DecValTok{37}\NormalTok{, }\DecValTok{35}\NormalTok{, }\DecValTok{36}\NormalTok{),}
                        \AttributeTok{gender =}\NormalTok{ genders,}
                        
                \AttributeTok{row.names =} \FunctionTok{colnames}\NormalTok{(genesMatrix))}
  
\NormalTok{  dataset }\OtherTok{\textless{}{-}} \FunctionTok{MultiAssayExperiment}\NormalTok{(}\FunctionTok{ExperimentList}\NormalTok{(}\AttributeTok{RNA =}\NormalTok{ genesMatrix, }\AttributeTok{RNA2 =}\NormalTok{ genesMatrix2, }\AttributeTok{RNA3 =}\NormalTok{ genesMatrix3), clinical)}

  
\DocumentationTok{\#\#\#\#\#\#\#\#\#\#\#\#\#\#\#\#\#\#\#\#\#\#\#\#\#\#\#\#\#\#\#\#\#\#\#\#\#\#\#\#\#\#\#\#\#\#\#\#\#\#\#\#\#\#\#\#\#\#\#\#\#\#\#\#\#\#\#\#\#\#\#\#\#\#\#\#\#\#\#\#}
\end{Highlighting}
\end{Shaded}

\hypertarget{mock-inputs}{%
\subsection{Mock Inputs}\label{mock-inputs}}

Below is where the user can modify parameters of ClassifyR's runTest/s()
function

New features include

\begin{itemize}
\item
  First tier fixing
\item
  Final tier retain-all
\end{itemize}

To be implemented

\begin{itemize}
\item
  Multiple Accuracy-threshold outputs
\item
  Adjustable training params
\item
  Parameter bounds (E.g. only display sequences \textless{} \$x)
\item
  Add Parameters (E.g. time, risk)
\item
  Extensive error checking of inputs
\item
  runTest() or runTests()
\end{itemize}

\begin{Shaded}
\begin{Highlighting}[]
\CommentTok{\#Mock Inputs}
  
\DocumentationTok{\#\#\#\#\#\#\#\#\#\#\#\#\#\#\#\#\#\#\#\#\#\#\#\#\#\#\#\#\#\#\#\#\#\#\#\#\#\#\#\#\#\#\#\#\#\#\#\#\#\#\#\#\#\#\#\#\#\#\#\#\#\#\#\#\#\#\#\#\#\#\#\#\#\#\#\#\#\#\#}

\CommentTok{\#Main MAE                                   }
\NormalTok{MultiAssayExperiment }\OtherTok{=}\NormalTok{ dataset}

\CommentTok{\#classificationNames {-} to be checked with classifyR docs}
\NormalTok{modelList }\OtherTok{=} \FunctionTok{list}\NormalTok{(}\StringTok{"Easy{-}Hard"}\NormalTok{, }\StringTok{"Easy{-}Hard"}\NormalTok{, }\StringTok{"Easy{-}Hard"}\NormalTok{) }

\CommentTok{\#Not sure if this will be needed for runTests}
\NormalTok{rsmpList }\OtherTok{=} \FunctionTok{list}\NormalTok{(}\StringTok{"rcv"}\NormalTok{, }\StringTok{"rcv"}\NormalTok{, }\StringTok{"boot"}\NormalTok{)}

\CommentTok{\#List of tiers}
\CommentTok{\#Can probably use names(experiments(measurements)) + clinical for tiers}
\NormalTok{tierList }\OtherTok{=} \FunctionTok{list}\NormalTok{(}\StringTok{"RNA1"}\NormalTok{, }\StringTok{"RNA2"}\NormalTok{, }\StringTok{"RNA3"}\NormalTok{) }

\CommentTok{\#Sample specific error cutoffs (E.g. x | if the sample specific error of cross validated prediction is \textgreater{} x, the sample will \textquotesingle{}progress\textquotesingle{}to the next tier. Lower cutoff = higher retention generally)}
\NormalTok{ssercutoffList }\OtherTok{=} \FunctionTok{c}\NormalTok{(}\FloatTok{0.4}\NormalTok{, }\FloatTok{0.4}\NormalTok{, }\FloatTok{0.4}\NormalTok{)}
\CommentTok{\#Multicutoffs = c(c(0.1, 0.1, 0.1),c(0.2, 0.2, 0.2),c(0.3, 0.3, 0.3) ) {-}\textgreater{} to be implemented for multi{-}threshold analysis}

\CommentTok{\#Tier costs}
\NormalTok{tierUnitCosts }\OtherTok{=} \FunctionTok{c}\NormalTok{(}\DecValTok{100}\NormalTok{, }\DecValTok{500}\NormalTok{, }\DecValTok{1000}\NormalTok{)}

\CommentTok{\#Which tier to fix as first in Tree? (If any)}
\CommentTok{\#fixedTier = "Clinical"}
\CommentTok{\# fixedTier = "RNA1"}
\NormalTok{fixedTier }\OtherTok{=} \ConstantTok{NULL}

\CommentTok{\#Model parameters?}
\NormalTok{selParams }\OtherTok{\textless{}{-}} \FunctionTok{SelectParams}\NormalTok{(}\AttributeTok{featureSelection =}\NormalTok{ differentMeansSelection, }\AttributeTok{selectionName =} \StringTok{"Difference in Means"}\NormalTok{,}
                            \AttributeTok{resubstituteParams =} \FunctionTok{ResubstituteParams}\NormalTok{(}\DecValTok{1}\SpecialCharTok{:}\DecValTok{10}\NormalTok{, }\StringTok{"balanced error"}\NormalTok{, }\StringTok{"lower"}\NormalTok{))}

\CommentTok{\#runtest() or runtests()}
\NormalTok{runtestorruntests }\OtherTok{=} \StringTok{"runtest"}

\CommentTok{\#name of classifier column in dataset}
\NormalTok{classes }\OtherTok{=} \StringTok{"class"}

\CommentTok{\#other params}
\NormalTok{params }\OtherTok{=} \FunctionTok{list}\NormalTok{(}\FunctionTok{SelectParams}\NormalTok{(), }\FunctionTok{TrainParams}\NormalTok{(), }\FunctionTok{PredictParams}\NormalTok{())}

\CommentTok{\#leave k out}
\NormalTok{leave }\OtherTok{=} \DecValTok{2}

\CommentTok{\#what percentage for training dataset (Not currently used)}
\NormalTok{percent}\OtherTok{=}\DecValTok{25}

\CommentTok{\#Minimum overlap}
\NormalTok{minimumOverlapPercent }\OtherTok{=} \DecValTok{80}

\CommentTok{\#Validation method}
\NormalTok{validation }\OtherTok{=} \FunctionTok{c}\NormalTok{(}\StringTok{"permute"}\NormalTok{, }\StringTok{"leaveOut"}\NormalTok{, }\StringTok{"fold"}\NormalTok{)}

\CommentTok{\#parallelization info}
\NormalTok{parallelParams }\OtherTok{=} \FunctionTok{bpparam}\NormalTok{()}

\CommentTok{\#EasyHard Params}
\NormalTok{easyDatasetID }\OtherTok{=} \StringTok{"clinical"}
\NormalTok{hardDatasetID }\OtherTok{=} \FunctionTok{names}\NormalTok{(MultiAssayExperiment)[}\DecValTok{1}\NormalTok{]}
\NormalTok{easyClassifierParams }\OtherTok{=} \FunctionTok{list}\NormalTok{(}\AttributeTok{minCardinality =} \DecValTok{2}\NormalTok{, }\AttributeTok{minPurity =} \FloatTok{0.9}\NormalTok{)}
\NormalTok{hardClassifierParams }\OtherTok{=} \FunctionTok{list}\NormalTok{(selParams, }\FunctionTok{TrainParams}\NormalTok{(), }\FunctionTok{PredictParams}\NormalTok{())}
  

\NormalTok{featureSets }\OtherTok{=} \ConstantTok{NULL}
\NormalTok{metaFeatures }\OtherTok{=} \ConstantTok{NULL}
\NormalTok{datasetName }\OtherTok{=} \StringTok{"Test Data"}


\NormalTok{classificationName }\OtherTok{=} \StringTok{"Easy{-}Hard"}

\NormalTok{training }\OtherTok{=} \DecValTok{1}\SpecialCharTok{:}\DecValTok{10}
\NormalTok{testing }\OtherTok{=} \DecValTok{1}\SpecialCharTok{:}\DecValTok{10}

\CommentTok{\#Cross Validation inputs}
\NormalTok{k}\OtherTok{=}\DecValTok{5}
\NormalTok{permutations}\OtherTok{=}\DecValTok{100}
\NormalTok{rsmp}\OtherTok{=}\DecValTok{20}


\NormalTok{seed}\OtherTok{=}\DecValTok{1}
\NormalTok{verbose}\OtherTok{=}\DecValTok{3}
  
\DocumentationTok{\#\#\#\#\#\#\#\#\#\#\#\#\#\#\#\#\#\#\#\#\#\#\#\#\#\#\#\#\#\#\#\#\#\#\#\#\#\#\#\#\#\#\#\#\#\#\#\#\#\#\#\#\#\#\#\#\#\#\#\#\#\#\#\#\#\#\#\#\#\#\#\#\#\#\#\#\#\#\#}
\end{Highlighting}
\end{Shaded}

\hypertarget{main-code}{%
\subsection{Main Code}\label{main-code}}

\begin{verbatim}
## [1] "(1) RNA1-RNA2-RNA3: <RNA1>"
\end{verbatim}

\begin{verbatim}
## Warning: package 'sparsediscrim' was built under R version 4.0.5
\end{verbatim}

\begin{verbatim}
## [1] "    Total = 10"
## [1] "    Processed = 10 (3 retained, 7 to progress to next tier)"
## [1] "    Not processed = 0"
## [1] "(1) RNA1-RNA2-RNA3: <RNA2>"
## [1] "    Total = 7"
## [1] "    Processed = 7 (3 retained, 4 to progress to next tier)"
## [1] "    Not processed = 0"
## [1] "(1) RNA1-RNA2-RNA3: <RNA3>"
## [1] "    Total = 4"
## [1] "    Processed = 4 (4 retained, 0 to progress to next tier)"
## [1] "    Not processed = 0"
## [1] "(2) RNA1-RNA3-RNA2: <RNA1>"
## [1] "    Total = 10"
## [1] "    Processed = 10 (3 retained, 7 to progress to next tier)"
## [1] "    Not processed = 0"
## [1] "(2) RNA1-RNA3-RNA2: <RNA3>"
## [1] "    Total = 7"
## [1] "    Processed = 7 (3 retained, 4 to progress to next tier)"
## [1] "    Not processed = 0"
## [1] "(2) RNA1-RNA3-RNA2: <RNA2>"
## [1] "    Total = 4"
## [1] "    Processed = 4 (4 retained, 0 to progress to next tier)"
## [1] "    Not processed = 0"
## [1] "(3) RNA2-RNA1-RNA3: <RNA2>"
## [1] "    Total = 10"
## [1] "    Processed = 10 (3 retained, 7 to progress to next tier)"
## [1] "    Not processed = 0"
## [1] "(3) RNA2-RNA1-RNA3: <RNA1>"
## [1] "    Total = 7"
## [1] "    Processed = 7 (3 retained, 4 to progress to next tier)"
## [1] "    Not processed = 0"
## [1] "(3) RNA2-RNA1-RNA3: <RNA3>"
## [1] "    Total = 4"
## [1] "    Processed = 4 (4 retained, 0 to progress to next tier)"
## [1] "    Not processed = 0"
## [1] "(4) RNA2-RNA3-RNA1: <RNA2>"
## [1] "    Total = 10"
## [1] "    Processed = 10 (3 retained, 7 to progress to next tier)"
## [1] "    Not processed = 0"
## [1] "(4) RNA2-RNA3-RNA1: <RNA3>"
## [1] "    Total = 7"
## [1] "    Processed = 7 (3 retained, 4 to progress to next tier)"
## [1] "    Not processed = 0"
## [1] "(4) RNA2-RNA3-RNA1: <RNA1>"
## [1] "    Total = 4"
## [1] "    Processed = 4 (4 retained, 0 to progress to next tier)"
## [1] "    Not processed = 0"
## [1] "(5) RNA3-RNA1-RNA2: <RNA3>"
## [1] "    Total = 10"
## [1] "    Processed = 10 (3 retained, 7 to progress to next tier)"
## [1] "    Not processed = 0"
## [1] "(5) RNA3-RNA1-RNA2: <RNA1>"
## [1] "    Total = 7"
## [1] "    Processed = 7 (3 retained, 4 to progress to next tier)"
## [1] "    Not processed = 0"
## [1] "(5) RNA3-RNA1-RNA2: <RNA2>"
## [1] "    Total = 4"
## [1] "    Processed = 4 (4 retained, 0 to progress to next tier)"
## [1] "    Not processed = 0"
## [1] "(6) RNA3-RNA2-RNA1: <RNA3>"
## [1] "    Total = 10"
## [1] "    Processed = 10 (3 retained, 7 to progress to next tier)"
## [1] "    Not processed = 0"
## [1] "(6) RNA3-RNA2-RNA1: <RNA2>"
## [1] "    Total = 7"
## [1] "    Processed = 7 (3 retained, 4 to progress to next tier)"
## [1] "    Not processed = 0"
## [1] "(6) RNA3-RNA2-RNA1: <RNA1>"
## [1] "    Total = 4"
## [1] "    Processed = 4 (4 retained, 0 to progress to next tier)"
## [1] "    Not processed = 0"
\end{verbatim}

\hypertarget{plots}{%
\subsection{Plots}\label{plots}}

Updates

\begin{itemize}
\tightlist
\item
  Accuracy instead of Error
\item
  Changes axis on bubble plot (from y = retention to y = cost)
\end{itemize}

New plots

\begin{itemize}
\tightlist
\item
  Tree plot with numbers of samples progressed represented as edge lines
\end{itemize}

To implement

\begin{itemize}
\tightlist
\item
  Table -\textgreater{} remove proportion retained but keep total
  accuracy?
\item
  Tree plot colouring
\end{itemize}

\begin{verbatim}
## PhantomJS not found. You can install it with webshot::install_phantomjs(). If it is installed, please make sure the phantomjs executable can be found via the PATH variable.
\end{verbatim}

\includegraphics{MTDT_GENERAL_OUTPUT_files/figure-latex/unnamed-chunk-3-2.pdf}

\begin{verbatim}
##                     levelName counter
## 1 RNA1                             NA
## 2  ¦--Retained (RNA1)               3
## 3  °--To progress (RNA1)            7
## 4      ¦--Retained (RNA2)           3
## 5      °--To progress (RNA2)        4
## 6          °--Retained (RNA3)       4
\end{verbatim}

\includegraphics{MTDT_GENERAL_OUTPUT_files/figure-latex/unnamed-chunk-3-3.pdf}

\begin{verbatim}
##                     levelName counter
## 1 RNA1                             NA
## 2  ¦--Retained (RNA1)               3
## 3  °--To progress (RNA1)            7
## 4      ¦--Retained (RNA3)           3
## 5      °--To progress (RNA3)        4
## 6          °--Retained (RNA2)       4
\end{verbatim}

\includegraphics{MTDT_GENERAL_OUTPUT_files/figure-latex/unnamed-chunk-3-4.pdf}

\begin{verbatim}
##                     levelName counter
## 1 RNA2                             NA
## 2  ¦--Retained (RNA2)               3
## 3  °--To progress (RNA2)            7
## 4      ¦--Retained (RNA1)           3
## 5      °--To progress (RNA1)        4
## 6          °--Retained (RNA3)       4
\end{verbatim}

\includegraphics{MTDT_GENERAL_OUTPUT_files/figure-latex/unnamed-chunk-3-5.pdf}

\begin{verbatim}
##                     levelName counter
## 1 RNA2                             NA
## 2  ¦--Retained (RNA2)               3
## 3  °--To progress (RNA2)            7
## 4      ¦--Retained (RNA3)           3
## 5      °--To progress (RNA3)        4
## 6          °--Retained (RNA1)       4
\end{verbatim}

\includegraphics{MTDT_GENERAL_OUTPUT_files/figure-latex/unnamed-chunk-3-6.pdf}

\begin{verbatim}
##                     levelName counter
## 1 RNA3                             NA
## 2  ¦--Retained (RNA3)               3
## 3  °--To progress (RNA3)            7
## 4      ¦--Retained (RNA1)           3
## 5      °--To progress (RNA1)        4
## 6          °--Retained (RNA2)       4
\end{verbatim}

\includegraphics{MTDT_GENERAL_OUTPUT_files/figure-latex/unnamed-chunk-3-7.pdf}

\begin{verbatim}
##                     levelName counter
## 1 RNA3                             NA
## 2  ¦--Retained (RNA3)               3
## 3  °--To progress (RNA3)            7
## 4      ¦--Retained (RNA2)           3
## 5      °--To progress (RNA2)        4
## 6          °--Retained (RNA1)       4
\end{verbatim}

\includegraphics{MTDT_GENERAL_OUTPUT_files/figure-latex/unnamed-chunk-3-8.pdf}

Note that the \texttt{echo\ =\ FALSE} parameter was added to the code
chunk to prevent printing of the R code that generated the plot.

\end{document}
